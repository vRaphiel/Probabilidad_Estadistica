\documentclass[a4paper]{article}
\usepackage{amsmath}
\setlength{\parskip}{1em}
\usepackage{ifthen}
\usepackage{amssymb}
\usepackage{multicol}
\usepackage{graphicx}
\usepackage[absolute]{textpos}

%\usepackage{xspace,rotating,calligra,dsfont,ifthen}
\usepackage{xspace,rotating,dsfont,ifthen}
\usepackage[spanish,activeacute]{babel}
\usepackage[utf8]{inputenc}
\usepackage{pgfpages}
\usepackage{pgf,pgfarrows,pgfnodes,pgfautomata,pgfheaps,xspace,dsfont}
\usepackage{listings}
\usepackage{multicol}


\makeatletter

\@ifclassloaded{beamer}{%
  \newcommand{\tocarEspacios}{%
    \addtolength{\leftskip}{4em}%
    \addtolength{\parindent}{-3em}%
  }%
}
{%
  \usepackage[top=1cm,bottom=2cm,left=1cm,right=1cm]{geometry}%
  \usepackage{color}%
  \newcommand{\tocarEspacios}{%
    \addtolength{\leftskip}{5em}%
    \addtolength{\parindent}{-3em}%
  }%
}

\newcommand{\encabezadoDeProc}[4]{%
  % Ponemos la palabrita problema en tt
%  \noindent%
  {\normalfont\bfseries\ttfamily proc}%
  % Ponemos el nombre del problema
  \ %
  {\normalfont\ttfamily #2}%
  \
  % Ponemos los parametros
  (#3)%
  \ifthenelse{\equal{#4}{}}{}{%
  \ =\ %
  % Ponemos el nombre del resultado
  {\normalfont\ttfamily #1}%
  % Por ultimo, va el tipo del resultado
  \ : #4}
}

\newcommand{\encabezadoDeTipo}[2]{%
  % Ponemos la palabrita tipo en tt
  {\normalfont\bfseries\ttfamily tipo}%
  % Ponemos el nombre del tipo
  \ %
  {\normalfont\ttfamily #2}%
  \ifthenelse{\equal{#1}{}}{}{$\langle$#1$\rangle$}
}

% Primero definiciones de cosas al estilo title, author, date

\def\materia#1{\gdef\@materia{#1}}
\def\@materia{No especifi\'o la materia}
\def\lamateria{\@materia}

\def\cuatrimestre#1{\gdef\@cuatrimestre{#1}}
\def\@cuatrimestre{No especifi\'o el cuatrimestre}
\def\elcuatrimestre{\@cuatrimestre}

\def\anio#1{\gdef\@anio{#1}}
\def\@anio{No especifi\'o el anio}
\def\elanio{\@anio}

\def\fecha#1{\gdef\@fecha{#1}}
\def\@fecha{\today}
\def\lafecha{\@fecha}

\def\nombre#1{\gdef\@nombre{#1}}
\def\@nombre{No especific'o el nombre}
\def\elnombre{\@nombre}

\def\practicas#1{\gdef\@practica{#1}}
\def\@practica{No especifi\'o el n\'umero de pr\'actica}
\def\lapractica{\@practica}


% Esta macro convierte el numero de cuatrimestre a palabras
\newcommand{\cuatrimestreLindo}{
  \ifthenelse{\equal{\elcuatrimestre}{1}}
  {Primer cuatrimestre}
  {\ifthenelse{\equal{\elcuatrimestre}{2}}
  {Segundo cuatrimestre}
  {Verano}}
}


\newcommand{\depto}{{UBA -- Facultad de Ciencias Exactas y Naturales --
      Departamento de Computaci\'on}}

\newcommand{\titulopractica}{
  \centerline{\depto}
  \vspace{1ex}
  \centerline{{\Large\lamateria}}
  \vspace{0.5ex}
  \centerline{\cuatrimestreLindo de \elanio}
  \vspace{2ex}
  \centerline{{\huge Pr\'actica \lapractica -- \elnombre}}
  \vspace{5ex}
  \arreglarincisos
  \newcounter{ejercicio}
  \newenvironment{ejercicio}{\stepcounter{ejercicio}\textbf{Ejercicio
      \theejercicio}%
    \renewcommand\@currentlabel{\theejercicio}%
  }{\vspace{0.2cm}}
}


\newcommand{\titulotp}{
  \centerline{\depto}
  \vspace{1ex}
  \centerline{{\Large\lamateria}}
  \vspace{0.5ex}
  \centerline{\cuatrimestreLindo de \elanio}
  \vspace{0.5ex}
  \centerline{\lafecha}
  \vspace{2ex}
  \centerline{{\huge\elnombre}}
  \vspace{5ex}
}


%practicas
\newcommand{\practica}[2]{%
    \title{Pr\'actica #1 \\ #2}
    \author{Algoritmos y Estructuras de Datos I}
    \date{Primer Cuatrimestre 2018}

    \maketitlepractica{#1}{#2}
}

\newcommand \maketitlepractica[2] {%
\begin{center}
\begin{tabular}{r cr}
 \begin{tabular}{c}
{\large\bf\textsf{\ Algoritmos y Estructuras de Datos I\ }}\\
Primer Cuatrimestre 2018\\
\title{\normalsize Gu\'ia Pr\'actica #1 \\ \textbf{#2}}\\
\@title
\end{tabular} &
\begin{tabular}{@{} p{1.6cm} @{}}
\includegraphics[width=1.6cm]{logodpt.jpg}
\end{tabular} &
\begin{tabular}{l @{}}
 \emph{Departamento de Computaci\'on} \\
 \emph{Facultad de Ciencias Exactas y Naturales} \\
 \emph{Universidad de Buenos Aires} \\
\end{tabular}
\end{tabular}
\end{center}

\bigskip
}


% Simbolos varios

\newcommand{\ent}{\ensuremath{\mathds{Z}}}
\newcommand{\float}{\ensuremath{\mathds{R}}}
\newcommand{\bool}{\ensuremath{\mathsf{Bool}}}
\newcommand{\True}{\ensuremath{\mathrm{true}}}
\newcommand{\False}{\ensuremath{\mathrm{false}}}
\newcommand{\Then}{\ensuremath{\rightarrow}}
\newcommand{\Iff}{\ensuremath{\leftrightarrow}}
\newcommand{\implica}{\ensuremath{\longrightarrow}}
\newcommand{\IfThenElse}[3]{\ensuremath{\mathsf{if}\ #1\ \mathsf{then}\ #2\ \mathsf{else}\ #3\ \mathsf{fi}}}
\newcommand{\In}{\textsf{in }}
\newcommand{\Out}{\textsf{out }}
\newcommand{\Inout}{\textsf{inout }}
\newcommand{\yLuego}{\land _L}
\newcommand{\oLuego}{\lor _L}
\newcommand{\implicaLuego}{\implica _L}
\newcommand{\existe}[3]{\ensuremath{(\exists #1:\ent) \ #2 \leq #1 < #3 \ }}
\newcommand{\paraTodo}[3]{\ensuremath{(\forall #1:\ent) \ #2 \leq #1 < #3 \ }}

% Símbolo para marcar los ejercicios importantes (estrellita)
\newcommand\importante{\raisebox{0.5pt}{\ensuremath{\bigstar}}}


\newcommand{\rango}[2]{[#1\twodots#2]}
\newcommand{\comp}[2]{[\,#1\,|\,#2\,]}

\newcommand{\rangoac}[2]{(#1\twodots#2]}
\newcommand{\rangoca}[2]{[#1\twodots#2)}
\newcommand{\rangoaa}[2]{(#1\twodots#2)}

%ejercicios
\newtheorem{exercise}{Ejercicio}
\newenvironment{ejercicio}[1][]{\begin{exercise}#1\rm}{\end{exercise} \vspace{0.2cm}}
\newenvironment{items}{\begin{enumerate}[a)]}{\end{enumerate}}
\newenvironment{subitems}{\begin{enumerate}[i)]}{\end{enumerate}}
\newcommand{\sugerencia}[1]{\noindent \textbf{Sugerencia:} #1}

\lstnewenvironment{code}{
    \lstset{% general command to set parameter(s)
        language=C++, basicstyle=\small\ttfamily, keywordstyle=\slshape,
        emph=[1]{tipo,usa}, emphstyle={[1]\sffamily\bfseries},
        morekeywords={tint,forn,forsn},
        basewidth={0.47em,0.40em},
        columns=fixed, fontadjust, resetmargins, xrightmargin=5pt, xleftmargin=15pt,
        flexiblecolumns=false, tabsize=2, breaklines, breakatwhitespace=false, extendedchars=true,
        numbers=left, numberstyle=\tiny, stepnumber=1, numbersep=9pt,
        frame=l, framesep=3pt,
    }
   \csname lst@SetFirstLabel\endcsname}
  {\csname lst@SaveFirstLabel\endcsname}


%tipos basicos
\newcommand{\rea}{\ensuremath{\mathsf{Float}}}
\newcommand{\cha}{\ensuremath{\mathsf{Char}}}
\newcommand{\str}{\ensuremath{\mathsf{String}}}

\newcommand{\mcd}{\mathrm{mcd}}
\newcommand{\prm}[1]{\ensuremath{\mathsf{prm}(#1)}}
\newcommand{\sgd}[1]{\ensuremath{\mathsf{sgd}(#1)}}

\newcommand{\tuple}[2]{\ensuremath{#1 \times #2}}

%listas
\newcommand{\TLista}[1]{\ensuremath{seq \langle #1\rangle}}
\newcommand{\lvacia}{\ensuremath{[\ ]}}
\newcommand{\lv}{\ensuremath{[\ ]}}
\newcommand{\longitud}[1]{\ensuremath{|#1|}}
\newcommand{\cons}[1]{\ensuremath{\mathsf{addFirst}}(#1)}
\newcommand{\indice}[1]{\ensuremath{\mathsf{indice}}(#1)}
\newcommand{\conc}[1]{\ensuremath{\mathsf{concat}}(#1)}
\newcommand{\cab}[1]{\ensuremath{\mathsf{head}}(#1)}
\newcommand{\cola}[1]{\ensuremath{\mathsf{tail}}(#1)}
\newcommand{\sub}[1]{\ensuremath{\mathsf{subseq}}(#1)}
\newcommand{\en}[1]{\ensuremath{\mathsf{en}}(#1)}
\newcommand{\cuenta}[2]{\mathsf{cuenta}\ensuremath{(#1, #2)}}
\newcommand{\suma}[1]{\mathsf{suma}(#1)}
\newcommand{\twodots}{\ensuremath{\mathrm{..}}}
\newcommand{\masmas}{\ensuremath{++}}
\newcommand{\matriz}[1]{\TLista{\TLista{#1}}}

% Acumulador
\newcommand{\acum}[1]{\ensuremath{\mathsf{acum}}(#1)}
\newcommand{\acumselec}[3]{\ensuremath{\mathrm{acum}(#1 |  #2, #3)}}

% \selector{variable}{dominio}
\newcommand{\selector}[2]{#1~\ensuremath{\leftarrow}~#2}
\newcommand{\selec}{\ensuremath{\leftarrow}}

\newcommand{\pred}[3]{%
    {\normalfont\bfseries\ttfamily pred }%
    {\normalfont\ttfamily #1}%
    \ifthenelse{\equal{#2}{}}{}{\ (#2) }%
    \{\ensuremath{#3}\}%
    {\normalfont\bfseries\,\par}%
  }

\newenvironment{proc}[4][res]{%
  % El parametro 1 (opcional) es el nombre del resultado
  % El parametro 2 es el nombre del problema
  % El parametro 3 son los parametros
  % El parametro 4 es el tipo del resultado
  % Preambulo del ambiente problema
  % Tenemos que definir los comandos requiere, asegura, modifica y aux
  \newcommand{\pre}[2][]{%
    {\normalfont\bfseries\ttfamily Pre}%
    \ifthenelse{\equal{##1}{}}{}{\ {\normalfont\ttfamily ##1} :}\ %
    \{\ensuremath{##2}\}%
    {\normalfont\bfseries\,\par}%
  }
  \newcommand{\post}[2][]{%
    {\normalfont\bfseries\ttfamily Post}%
    \ifthenelse{\equal{##1}{}}{}{\ {\normalfont\ttfamily ##1} :}\
    \{\ensuremath{##2}\}%
    {\normalfont\bfseries\,\par}%
  }
  \renewcommand{\aux}[4]{%
    {\normalfont\bfseries\ttfamily fun\ }%
    {\normalfont\ttfamily ##1}%
    \ifthenelse{\equal{##2}{}}{}{\ (##2)}\ : ##3\, = \ensuremath{##4}%
    {\normalfont\bfseries\,;\par}%
  }
  \newcommand{\res}{#1}
  \vspace{1ex}
  \noindent
  \encabezadoDeProc{#1}{#2}{#3}{#4}
  % Abrimos la llave
  \{\par%
  \tocarEspacios
}
% Ahora viene el cierre del ambiente problema
{
  % Cerramos la llave
  \noindent\}
  \vspace{1ex}
}


  \newcommand{\aux}[4]{%
    {\normalfont\bfseries\ttfamily fun\ }%
    {\normalfont\ttfamily #1}%
    \ifthenelse{\equal{#2}{}}{}{\ (#2)}\ : #3\, = \ensuremath{#4}%
    {\normalfont\bfseries\,;\par}%
  }


% \newcommand{\pre}[1]{\textsf{pre}\ensuremath{(#1)}}

\newcommand{\procnom}[1]{\textsf{#1}}
\newcommand{\procil}[3]{\textsf{proc #1}\ensuremath{(#2) = #3}}
\newcommand{\procilsinres}[2]{\textsf{proc #1}\ensuremath{(#2)}}
\newcommand{\preil}[2]{\textsf{Pre #1: }\ensuremath{#2}}
\newcommand{\postil}[2]{\textsf{Post #1: }\ensuremath{#2}}
\newcommand{\auxil}[2]{\textsf{fun }\ensuremath{#1 = #2}}
\newcommand{\auxilc}[4]{\textsf{fun }\ensuremath{#1( #2 ): #3 = #4}}
\newcommand{\auxnom}[1]{\textsf{fun }\ensuremath{#1}}
\newcommand{\auxpred}[3]{\textsf{pred }\ensuremath{#1( #2 ) \{ #3 \}}}

\newcommand{\comentario}[1]{{/*\ #1\ */}}

\newcommand{\nom}[1]{\ensuremath{\mathsf{#1}}}


% En las practicas/parciales usamos numeros arabigos para los ejercicios.
% Aca cambiamos los enumerate comunes para que usen letras y numeros
% romanos
\newcommand{\arreglarincisos}{%
  \renewcommand{\theenumi}{\alph{enumi}}
  \renewcommand{\theenumii}{\roman{enumii}}
  \renewcommand{\labelenumi}{\theenumi)}
  \renewcommand{\labelenumii}{\theenumii)}
}



%%%%%%%%%%%%%%%%%%%%%%%%%%%%%% PARCIAL %%%%%%%%%%%%%%%%%%%%%%%%
\let\@xa\expandafter
\newcommand{\tituloparcial}{\centerline{\depto -- \lamateria}
  \centerline{\elnombre -- \lafecha}%
  \setlength{\TPHorizModule}{10mm} % Fija las unidades de textpos
  \setlength{\TPVertModule}{\TPHorizModule} % Fija las unidades de
                                % textpos
  \arreglarincisos
  \newcounter{total}% Este contador va a guardar cuantos incisos hay
                    % en el parcial. Si un ejercicio no tiene incisos,
                    % cuenta como un inciso.
  \newcounter{contgrilla} % Para hacer ciclos
  \newcounter{columnainicial} % Se van a usar para los cline cuando un
  \newcounter{columnafinal}   % ejercicio tenga incisos.
  \newcommand{\primerafila}{}
  \newcommand{\segundafila}{}
  \newcommand{\rayitas}{} % Esto va a guardar los \cline de los
                          % ejercicios con incisos, asi queda mas bonito
  \newcommand{\anchodegrilla}{20} % Es para textpos
  \newcommand{\izquierda}{7} % Estos dos le dicen a textpos donde colocar
  \newcommand{\abajo}{2}     % la grilla
  \newcommand{\anchodecasilla}{0.4cm}
  \setcounter{columnainicial}{1}
  \setcounter{total}{0}
  \newcounter{ejercicio}
  \setcounter{ejercicio}{0}
  \renewenvironment{ejercicio}[1]
  {%
    \stepcounter{ejercicio}\textbf{\noindent Ejercicio \theejercicio. [##1
      puntos]}% Formato
    \renewcommand\@currentlabel{\theejercicio}% Esto es para las
                                % referencias
    \newcommand{\invariante}[2]{%
      {\normalfont\bfseries\ttfamily invariante}%
      \ ####1\hspace{1em}####2%
    }%
    \newcommand{\Proc}[5][result]{
      \encabezadoDeProc{####1}{####2}{####3}{####4}\hspace{1em}####5}%
  }% Aca se termina el principio del ejercicio
  {% Ahora viene el final
    % Esto suma la cantidad de incisos o 1 si no hubo ninguno
    \ifthenelse{\equal{\value{enumi}}{0}}
    {\addtocounter{total}{1}}
    {\addtocounter{total}{\value{enumi}}}
    \ifthenelse{\equal{\value{ejercicio}}{1}}{}
    {
      \g@addto@macro\primerafila{&} % Si no estoy en el primer ej.
      \g@addto@macro\segundafila{&}
    }
    \ifthenelse{\equal{\value{enumi}}{0}}
    {% No tiene incisos
      \g@addto@macro\primerafila{\multicolumn{1}{|c|}}
      \bgroup% avoid overwriting somebody else's value of \tmp@a
      \protected@edef\tmp@a{\theejercicio}% expand as far as we can
      \@xa\g@addto@macro\@xa\primerafila\@xa{\tmp@a}%
      \egroup% restore old value of \tmp@a, effect of \g@addto.. is

      \stepcounter{columnainicial}
    }
    {% Tiene incisos
      % Primero ponemos el encabezado
      \g@addto@macro\primerafila{\multicolumn}% Ahora el numero de items
      \bgroup% avoid overwriting somebody else's value of \tmp@a
      \protected@edef\tmp@a{\arabic{enumi}}% expand as far as we can
      \@xa\g@addto@macro\@xa\primerafila\@xa{\tmp@a}%
      \egroup% restore old value of \tmp@a, effect of \g@addto.. is
      % global
      % Ahora el formato
      \g@addto@macro\primerafila{{|c|}}%
      % Ahora el numero de ejercicio
      \bgroup% avoid overwriting somebody else's value of \tmp@a
      \protected@edef\tmp@a{\theejercicio}% expand as far as we can
      \@xa\g@addto@macro\@xa\primerafila\@xa{\tmp@a}%
      \egroup% restore old value of \tmp@a, effect of \g@addto.. is
      % global
      % Ahora armamos la segunda fila
      \g@addto@macro\segundafila{\multicolumn{1}{|c|}{a}}%
      \setcounter{contgrilla}{1}
      \whiledo{\value{contgrilla}<\value{enumi}}
      {%
        \stepcounter{contgrilla}
        \g@addto@macro\segundafila{&\multicolumn{1}{|c|}}
        \bgroup% avoid overwriting somebody else's value of \tmp@a
        \protected@edef\tmp@a{\alph{contgrilla}}% expand as far as we can
        \@xa\g@addto@macro\@xa\segundafila\@xa{\tmp@a}%
        \egroup% restore old value of \tmp@a, effect of \g@addto.. is
        % global
      }
      % Ahora armo las rayitas
      \setcounter{columnafinal}{\value{columnainicial}}
      \addtocounter{columnafinal}{-1}
      \addtocounter{columnafinal}{\value{enumi}}
      \bgroup% avoid overwriting somebody else's value of \tmp@a
      \protected@edef\tmp@a{\noexpand\cline{%
          \thecolumnainicial-\thecolumnafinal}}%
      \@xa\g@addto@macro\@xa\rayitas\@xa{\tmp@a}%
      \egroup% restore old value of \tmp@a, effect of \g@addto.. is
      \setcounter{columnainicial}{\value{columnafinal}}
      \stepcounter{columnainicial}
    }
    \setcounter{enumi}{0}%
    \vspace{0.2cm}%
  }%
  \newcommand{\tercerafila}{}
  \newcommand{\armartercerafila}{
    \setcounter{contgrilla}{1}
    \whiledo{\value{contgrilla}<\value{total}}
    {\stepcounter{contgrilla}\g@addto@macro\tercerafila{&}}
  }
  \newcommand{\grilla}{%
    \g@addto@macro\primerafila{&\textbf{TOTAL}}
    \g@addto@macro\segundafila{&}
    \g@addto@macro\tercerafila{&}
    \armartercerafila
    \ifthenelse{\equal{\value{total}}{\value{ejercicio}}}
    {% No hubo incisos
      \begin{textblock}{\anchodegrilla}(\izquierda,\abajo)
        \begin{tabular}{|*{\value{total}}{p{\anchodecasilla}|}c|}
          \hline
          \primerafila\\
          \hline
          \tercerafila\\
          \tercerafila\\
          \hline
        \end{tabular}
      \end{textblock}
    }
    {% Hubo incisos
      \begin{textblock}{\anchodegrilla}(\izquierda,\abajo)
        \begin{tabular}{|*{\value{total}}{p{\anchodecasilla}|}c|}
          \hline
          \primerafila\\
          \rayitas
          \segundafila\\
          \hline
          \tercerafila\\
          \tercerafila\\
          \hline
        \end{tabular}
      \end{textblock}
    }
  }%
  \vspace{0.4cm}
  \textbf{Nro. de orden:}

  \textbf{LU:}

  \textbf{Apellidos:}

  \textbf{Nombres:}
  \vspace{0.5cm}
}



% AMBIENTE CONSIGNAS
% Se usa en el TP para ir agregando las cosas que tienen que resolver
% los alumnos.
% Dentro del ambiente hay que usar \item para cada consigna

\newcounter{consigna}
\setcounter{consigna}{0}

\newenvironment{consignas}{%
  \newcommand{\consigna}{\stepcounter{consigna}\textbf{\theconsigna.}}%
  \renewcommand{\ejercicio}[1]{\item ##1 }
  \renewcommand{\proc}[5][result]{\item
    \encabezadoDeProc{##1}{##2}{##3}{##4}\hspace{1em}##5}%
  \newcommand{\invariante}[2]{\item%
    {\normalfont\bfseries\ttfamily invariante}%
    \ ##1\hspace{1em}##2%
  }
  \renewcommand{\aux}[4]{\item%
    {\normalfont\bfseries\ttfamily aux\ }%
    {\normalfont\ttfamily ##1}%
    \ifthenelse{\equal{##2}{}}{}{\ (##2)}\ : ##3 \hspace{1em}##4%
  }
  % Comienza la lista de consignas
  \begin{list}{\consigna}{%
      \setlength{\itemsep}{0.5em}%
      \setlength{\parsep}{0cm}%
    }
}%
{\end{list}}



% para decidir si usar && o ^
\newcommand{\y}[0]{\ensuremath{\land}}

% macros de correctitud
\newcommand{\semanticComment}[2]{#1 \ensuremath{#2};}
\newcommand{\namedSemanticComment}[3]{#1 #2: \ensuremath{#3};}


\newcommand{\local}[1]{\semanticComment{local}{#1}}

\newcommand{\vale}[1]{\semanticComment{vale}{#1}}
\newcommand{\valeN}[2]{\namedSemanticComment{vale}{#1}{#2}}
\newcommand{\impl}[1]{\semanticComment{implica}{#1}}
\newcommand{\implN}[2]{\namedSemanticComment{implica}{#1}{#2}}
\newcommand{\estado}[1]{\semanticComment{estado}{#1}}

\newcommand{\invarianteCN}[2]{\namedSemanticComment{invariante}{#1}{#2}}
\newcommand{\invarianteC}[1]{\semanticComment{invariante}{#1}}
\newcommand{\varianteCN}[2]{\namedSemanticComment{variante}{#1}{#2}}
\newcommand{\varianteC}[1]{\semanticComment{variante}{#1}}


\begin{document}

\section{Ejemplos de estimacion puntual}

\subsection{Ejercicio 1 - Distribucion Normal}
Se quiere conocer el peso medio de los paquetes de arroz producido por una fábrica. Para ello se toman 30 cajas
de arroz al azar y se las pesa. Se obtiene

\begin{equation*}
    \begin{matrix}
        0.96 & 0.97 & 1.12 & 1.16 & 1.03 & 0.95 & 0.91 & 0.87 & 0.96 & 1.04 \\
        0.77 & 0.99 & 0.84 & 1.08 & 1.12 & 0.78 & 0.95 & 0.93 & 1.09 & 0.92 \\
        1.00 & 0.92 & 1.02 & 0.90 & 0.87 & 0.85 & 1.03 & 1.04 & 0.92 & 1.07
    \end{matrix}
\end{equation*}

Supongamos que el peso de un paquete elegido al azar es una variable aleatoria $X \sim \mathcal{N}(\mu,\sigma^2)$

Al elegir $n$ paquetes tenemos: Sea $X_{1},X_{2},\dots,X_{n}$ i.i.d. $X_{i} \sim \mathcal{N}(\mu,\,\sigma^{2})$
Estimador de los momentos de $\mu$ de orden 1:
\begin{equation*}
    \frac{1}{n} \sum_{i=1}^n X_{i} = E_{\widehat{\mu}}(X) = \widehat{\mu}
\end{equation*}
Con estos datos la estimacion que se obtiene es $\widehat{\mu}_{obs} = 0.97$

\subsection{Ejercicio 2 - Distribucion Exponencial}
Una fabrica de lamparas sabe que el tiempo de vida, en dias, de las lamparas que fabrica, sigue una distribucion $Exp(\theta)$.
Obtener una formula para estimar $\theta$ a partir de una muestra aleatoria $X_{1}\dots X_{n}$

Antes de probar las lamparas no sabemos cuanto durara cada una. Asi la duracion de la primera puede ser considerada
una v.a $X_{1}$. la segunda una v.a $X_{2}$, etc. 

\begin{equation*}
    X_{1}, X_{2}, \dots, X_{n} \sim Exp(\theta)
\end{equation*}

Para hallar el estimador de momentos de $\theta$, hay que despejar $\widehat{\theta}$ de
\begin{equation*}
    \frac{1}{n} \sum_{i=1}^n X_{i} = E_{\widehat{\theta}}(X_{1})
\end{equation*}
\begin{equation*}
    \frac{1}{n} \sum_{i=1}^n X_{i} = \frac{1}{\mathcal{\theta}} \Rightarrow \mathcal{\theta} = \frac{1}{\widehat{X}}
\end{equation*}

\subsection{Ejercicio 3 - Distribucion uniforme}
\begin{equation*}
    X_{1}, X_{2}, \dots , X_{n} \sim \mathcal{U}{0, \theta}
\end{equation*}
Hay que despejar $\theta$ de
\begin{equation*}
    \frac{1}{n} \sum_{i=1}^n X_{i} = E_{\widehat{\theta}}(X_{1})
\end{equation*}
\begin{equation*}
    \frac{1}{n} \sum_{i=1}^n X_{i} = \frac{\widehat{\theta}}{2} \Rightarrow \mathcal{\theta} = 2\frac{1}{n}\sum_{i=1}^{n} X_{i}
\end{equation*}

\subsection{Ejercicio 4 - Estimacion de ambos parametros de la normal}
Supongamos que tenemos una muestra aleatoria $X_{1}, X_{2},\dots,X_{n} \sim \mathcal{N}(\mu, \sigma^2)$ \\
Se tiene que $E_{\mu, \sigma^2}(X) = \mu$ y $E_{\mu, \sigma^2}(X^2) = \mu^2 + \sigma^2$
Para encontrar el estimador de momentos de $\mu$ y $\sigma$ hay que resolver el sistema

\begin{equation*}
    \frac{1}{n} \sum_{i=1}^n X_{i} = E_{\widehat{\mu}, \widehat{\sigma}^2}(X_{1}) = \widehat{\mu}
\end{equation*}
\begin{equation*}
    \frac{1}{n} \sum_{i=1}^n X_{i}^2 = E_{\widehat{\mu}, \widehat{\sigma}^2}(X_{1}^2) = \widehat{\mu}^2 + \widehat{\sigma}^2
\end{equation*}

\begin{equation*}
    \begin{cases}
        \frac{1}{n} \sum_{i=1}^n X_{i} = \widehat{\mu}
        \\
        \frac{1}{n} \sum_{i=1}^n X_{i}^2 = \widehat{\mu}^2 + \widehat{\sigma}^2
        \end{cases}
\end{equation*}
\begin{equation*}
    \widehat{\mu} = \frac{1}{n} \sum_{i=1}^n X_{i}
\end{equation*}
\begin{equation*}
    \widehat{\sigma}^2 = \frac{1}{n} \sum_{i=1}^n X_{i}^2 - (\frac{1}{n} \sum_{i=1}^n X_{i})^2
\end{equation*}

\section{Ejemplos de estimacion puntual - Verosimilitud}
\subsection{Ejercicio 1 - $\mathcal{E}(\lambda):f(x,\lambda) = \lambda e^{-x\lambda}\mathcal{I}_{(0, \infty)}(x)$}
$X_{1},\dots,X_{n}$ v.a. i.i.d. $X_{i} \sim \mathcal{E}(\lambda), \lambda > 0$
\begin{equation*}
    L(\lambda; x) = \prod_{i=1}^n f(x_{i}, \lambda) = \prod_{i=1}^n \lambda e^{-x_{i}\lambda}\mathcal{I}_{(0,\infty)}(x_{i})
\end{equation*}
Si $x_{i}\geq 0 \forall i$
\begin{equation*}
    L(\lambda; x) = \lambda e^{-x_{i}\sum_{i=1}^n x_{i}}
\end{equation*}
Si consideramos $\log L$ resulta
\begin{equation*}
    l(\lambda;x) = n \log(\lambda)-\lambda \sum_{i=1}^n x_{i}
\end{equation*}
Derivando e igualando a 0 queda \\
$\frac{n}{\lambda} - \sum_{i=1}^n x_{i} = 0 \Rightarrow$ punto critico es $\frac{1}{\bar{x}_{n}}$, ver que maximiza \\
$\Rightarrow \widehat{\lambda} = \frac{1}{\bar{X}_{n}}$

\subsection{Ejercicio 2 - $X_{1},\dots,X_{n}$ v.a. i.i.d. $X_{i} \sim \mathcal{N} (\mu, 9)$, $f(x,\mu, 9) = \frac{1}{\sqrt{2\pi}}\frac{1}{3}e^{-\frac{1}{2}\frac{(x-\mu)^2}{9}}$ }
\begin{equation*}
    L(\mu, 9; x) = \prod_{i=1}^n f(x_{i}, \mu, 9) = \prod_{i=1}^n \frac{1}{\sqrt(2\pi)}\frac{1}{3}e^{-\frac{1}{2}\frac{(x_{i}-\mu)^2}{9}}
\end{equation*}
\begin{equation*}
    L(\mu, 9; x) = (\frac{1}{\sqrt(2\pi)})^n(\frac{1}{3})^n e^{-\frac{1}{2}\frac{\sum_{i=1}^n(x_{i}-\mu)^2}{9}}
\end{equation*}
Tomemos logatirmo
\begin{equation*}
    l(\mu, 9; x) = cte - \frac{1}{2}\frac{\sum_{i=1}^n(x_{i}-\mu)^2}{9}
\end{equation*}
Maximizar a $l(\mu, 9; x)$ como funcion de $\mu$ equivale a minimizar
\begin{equation*}
    h(\mu) = \sum_{i=1}^n (x_{i} - \mu)^2
\end{equation*}
Un par de clases aatras vimos que $h(\mu)$ se minimiza en $\bar{x}_{n}$
\begin{equation*}
    EMV de \ \mu : \widehat{\mu} = \bar{X}_{n}
\end{equation*}
\begin{equation*}
    L(\mu, 9; x) = \prod_{i=1}^n f(x_{i}, \mu, \sigma^2) = \prod_{i=1}^n \frac{1}{\sqrt(2\pi\sigma^2)}e^{-\frac{(x_{i}-\mu)^2}{2\sigma^2}}
\end{equation*}
\begin{equation*}
    L(\mu, 9; x) = \prod_{i=1}^n f(x_{i}, \mu, \sigma^2) = \prod_{i=1}^n \frac{1}{\sqrt(2\pi\sigma^2)}e^{-\sum_{i=1}^n \frac{(x_{i}-\mu)^2}{2\sigma^2}}
\end{equation*}
Tomando logatirmo y resolviendo las ecuaciones
\begin{equation*}
    \frac{\partial l (\mu, \sigma^2;x)}{\partial \mu} = 0 \ y \ \frac{\partial l (\mu, \sigma^2;x)}{\partial \sigma^2} = 0
\end{equation*}
se obtiene que los EMV de $\mu$ y $\sigma^2$ section
\begin{equation*}
    \widehat{\mu} = \bar{X}_{n} \ \ \ \ \widehat{\sigma}^2 = \frac{\sum_{i=1}^n ((X_{i} - \bar{X}_n))^2}{n}
\end{equation*}
Los estimadores de $\mu$ y $\sigma$ coinciden con los estimadores de momentos

\subsection{Ejercicio 2 - $\mathcal{U}(0,\theta):f(x,\theta) = \frac{1}{\theta}\mathcal{I}_{(0,\theta)}(x)$}
$X_{1},\dots,X_{n}$ v.a. i.i.d. $X_{i} \sim \mathcal{U}(0,\theta)$
\begin{equation*}
    f(x_{1},\dots,x_{n}, \theta) = \prod_{i=1}^n \frac{1}{\theta}I_{(0,\theta)}(x_{i}) = \frac{1}{\theta^n}\prod_{i=1}^n I_{(0,\theta)}(x_{i})
\end{equation*}
La indicadora vale 1 cuando todas las indicadoras valgan 1, de lo contrario, es 0
Por ende, la conjunta puede tomar 2 valores:
\begin{equation*}
    \begin{cases}
        \frac{1}{\theta^n} \ si \ 0<x_{i}<\theta \  \forall i
        \\
        0 \ en \ otro \ caso
        \end{cases}
\end{equation*}
\begin{equation*}
    \begin{cases}
        \frac{1}{\theta^n} \ si \ \theta > \max(x_{i})
        \\
        0 \ en \ otro \ caso
        \end{cases}
\end{equation*}

\pagebreak

\section{Propiedades de estimacion}
\subsection{Ejercicio 1 - Distribucion Uniforme}
\begin{itemize}
    \item $X_i \sim \mathcal{U}(0, \theta)$
    \item $\widehat{\theta} = 2\bar{X}_n$ (Estimador de momentos)
    \item $\widehat{\theta} = max{X_{1},\dots,X_{n}}$ (Estimador de maxima Verosimilitud)
    \item Calcule la esperanza y varianza de cada estimador
\end{itemize}

Esperanza del estimador de momentos
Sea $X_{1},X_{2},\dots,X_{n}$ una m.a de una distribucion $\mathcal{U}(0, \theta)$. El estimador de momentos de $\theta$ es $\widehat{\theta} = 2\bar{X}$

\begin{equation*}
    E_{\theta}(\widehat{\theta}) = 2E_{\theta}\bar{X} = 2\frac{\theta}{2} = \theta \ \ \forall \theta
\end{equation*}

Recordar

Sea $X_{1},X_{2},\dots,X_{n}$ una m.a de una distribucion $\mathcal{U}(0, \theta)$
\begin{itemize}
    \item El EMV de $\theta = max_{1\leq i \leq n}(X_{i})$
    \item La fda es $\widehat{\theta}$ es
        \begin{equation*}
            F_{\widehat{\theta}}(u) = (F_{X_{i}}(u))^n =       
            \begin{cases}
                0 \ si \ u \leq \theta
                \\
                (\frac{u}{\theta})^n \ si \ 0<u<\theta
                \\
                1 \ si \ u \geq \theta
            \end{cases}
        \end{equation*}
    \item La densidad de $\widehat{\theta}$ es
        \begin{equation*}
            f_{\widehat{\theta}}(u) = n\frac{u}{\theta}^{n-1} \frac{1}{\theta} I_{(0,\theta)}(u)
        \end{equation*}
\end{itemize}

\begin{equation*}
    E_{\theta}(\widehat{\theta}) = \int_{0}^{\theta} un(\frac{u}{\theta})^{n-1} \frac{1}{\theta} du = 
    \\
    \frac{n}{\theta^n}\int_{0}^{\theta} u^n du = \frac{n}{\theta^n}\frac{u^{n+1}}{n+1}\Biggr|_{0}^{\theta} = \frac{n}{n+1}\theta
\end{equation*}

Varianza del estimador de momentos

Sea $X_{1},X_{2},\dots,X_{n}$ una m.a de una distribucion $\mathcal{U}(0, \theta)$ y $\widehat{\theta}=2\bar{X}$ el estimador de momentos de $\theta$
\begin{equation*}
    V_{\theta}(\widehat{\theta}) = 4V_{\theta}(\bar{X}) = 4 \frac{\theta^2/12}{n}
\end{equation*}

Varianza del estimador de maxima Verosimilitud
Recordar la densidad del EMV
\begin{gather*}
    f_{\theta}(u) = n\frac{u}{\theta}^{n-1}\frac{1}{\theta}I_{(0,\theta)}(u)
    \\
    E_{\theta}(\widehat{\theta}^2) = \int_{0}^{\theta} u^2n \frac{u}{\theta}^{n-1} \frac{1}{\theta}du = \frac{n}{\theta^n}\int_{0}^{\theta}u^{n+1} du
    \\
    \frac{n}{\theta^n}\frac{u^{n+2}}{n+2}\Biggr|_{0}^{\theta} = \frac{n}{n+2}\theta^2
\end{gather*}
Entonces,
\begin{gather*}
    V_{\theta}(\widehat{\theta}) = \frac{n}{n+2}\theta^2 - (\frac{n}{n+1})^2\theta^2=(\frac{n}{n+2} - \frac{n^2}{(n+1)^2})\theta^2
    \\
    \frac{n}{(n+2)(n+1)^2}\theta^2
\end{gather*}

\subsection{Ejercicio 2 - Sesgo de los EMV de $\mu$ y $\sigma^2$ en el caso normal}
Sea $X_{1},X_{2},\dots,X_{n}$ una m.a de una distribucion $\mathcal{N}(\mu, \sigma^2)$. Los EMV de $\mu$ y $\sigma^2$ section
\begin{equation*}
    \widehat{\mu} = \bar{X} \ \ \ \ \widehat{\sigma}^2 = \frac{\sum_{i=1}^n (X_{i} - \bar{X})^2}{n}
\end{equation*}

\begin{itemize}
    \item El estimador de $\mu$ es insesgado pues $E_{\mu, \sigma^2}(\widehat{\mu}) = \mu \forall\mu$,
    \item El estimador de $\sigma^2$ es sesgado pero es asintoticamente insesgado pues:
\end{itemize}
\begin{gather*}
    E_{\mu, \sigma^2}(\widehat{\sigma}^2) = E_{\mu, \sigma^2}(\frac{\sum_{i=1}^n (X_{i} - \bar{X})^2}{n})
    \\
    = \frac{1}{n}E_{\mu, \sigma^2}(\sum_{i=1}^n (X_{i}^2 - 2X_{i}\bar{X} + \bar{X}^2))
    \\
    = \frac{1}{n}E_{\mu, \sigma^2}(\sum_{i=1}^n X_{i}^2 - 2\bar{X}\sum_{i=1}^n X_{i} + n\bar{X}^2)
    \\
    = \frac{1}{n}E_{\mu, \sigma^2}(\sum_{i=1}^n X_{i}^2 - 2n\bar{X}^2 + n\bar{X}^2)
    \\
    = \frac{1}{n}E_{\mu, \sigma^2}(\sum_{i=1}^n X_{i}^2 - n\bar{X}^2)
    \\
    = \frac{1}{n}E_{\mu, \sigma^2}(\sum_{i=1}^n X_{i}^2) - E_{\mu, \sigma^2}(\bar{X}^2)
    \\
    = \frac{n}{n}E_{\mu, \sigma^2}(X_{1}^2) - E_{\mu, \sigma^2}(\bar{X}^2)
    \\
    = [ V_{\mu, \sigma^2}(X_{1}) + (E_{\mu, \sigma^2}(X_{1}))^2] - [ V_{\mu, \sigma^2}(\bar{X}) + (E_{\mu, \sigma^2}(\bar{X_{1}}))^2]
    \\
    = \sigma^2 + \mu^2 - \frac{\sigma^2}{n} - \mu^2
    \\
    = \frac{n-1}{n}\sigma^2
\end{gather*}

\subsection{Ejemplo - Consistencia de la media muestral}
Sea $X_{1},X_{2},\dots,X_{n}$ una m.a de una distribucion con $E(X_{i})=\mu$ y $V(X_{i}) = \sigma^2 < \infty$ 
Sea $\widehat{\mu} = \bar{X}$
\begin{itemize}
    \item $E(\bar{X}) = \mu$
    \item $V(\bar{X}) = \frac{\sigma^2}{n}$
\end{itemize}
Entonces $\widehat{\mu}$ es un estimador consistente de $\mu$

\subsection{Ejemplo - Consistencia de los estimadores de $\theta$ en la $\mathcal{U}[0,\theta]$}
Sea $X_{1},X_{2},\dots,X_{n}$ una m.a de una distribucion $\mathcal{U}(0, \theta)$. Vimos que el EMV de $\theta, \widehat{\theta} = max_{1\leq i \leq n}(X_{i})$ y 
\begin{itemize}
    \item $E_{\theta}(\widehat{\theta}) = \frac{n}{n+1}\theta \Rightarrow \widehat{\theta}$ es asintóticamente insesgado
    \item $V_{\theta}(\widehat{\theta}) = \frac{n}{(n+2)(n+1)^2}\theta^2 \rightarrow_{n\rightarrow\infty}0$
\end{itemize}
Entonces el $\widehat{\theta}$ es consistente

\subsection{Ejemplo - Consistencia de la varianza muestral}
Sea $X_{1},X_{2},\dots,X_{n}$ una m.a de una con $E(X_{i}) = \mu$ y $V(X_{i}) = \sigma^2 < \infty$ entonces la varianza muestral $S^2$ es un estimador
consistente de la varianza poblacional
\begin{gather*}
    S^2 = \frac{\sum_{i=1}^n (X_{i} - \bar{X})^2}{n-1} = \frac{1}{n-1}(\sum_{i=1}^n X_{i}^2 - n\bar{X}^2)
    \\
    = \frac{n}{n-1}(\frac{\sum_{i=1}^n X_{i}^2}{n} - \bar{X}^2)
\end{gather*}

Por la Ley de los Grandes Numeros $\bar{X} \xrightarrow[]{p} \mu$, entonces por la propiedad 4,
\begin{equation*}
    \bar{X}^2 \xrightarrow[p]{} \mu^2
\end{equation*}

Por otra parte, aplicando nuevamente la Ley de los grandes Numeros
\begin{equation*}
    \frac{\sum_{i=1}^n X_{i}^2}{n} \xrightarrow[]{p} E_{\mu, \sigma^2}(X^2) = V_{\mu, \sigma^2}(X) + [E_{\mu, \sigma^2}(X)]^2 = \sigma^2 + \mu^2
\end{equation*}
Como ademas $\frac{n}{n-1} \rightarrow 1$, se obtiene
\begin{equation*}
    S_{x}^2 = \frac{n}{n-1}(\frac{\sum_{i=1}^n X_{i}^2}{n} - \bar{X}^2) \xrightarrow[]{p} \sigma^2 + \mu^2 -\mu^2 = \sigma^2
\end{equation*}
y por lo tanto la varianza muestral es un estimador consistente de $\sigma^2$

\end{document}
