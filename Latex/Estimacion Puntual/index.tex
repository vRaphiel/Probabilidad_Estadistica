\documentclass[a4paper]{article}
\usepackage{amsmath}
\setlength{\parskip}{1em}
\input{Algo1Macros}

\begin{document}

\section{Ejemplos de estimacion puntual}

\subsection{Ejercicio 1 - Distribucion Normal}
Se quiere conocer el peso medio de los paquetes de arroz producido por una fábrica. Para ello se toman 30 cajas
de arroz al azar y se las pesa. Se obtiene

\begin{equation*}
    \begin{matrix}
        0.96 & 0.97 & 1.12 & 1.16 & 1.03 & 0.95 & 0.91 & 0.87 & 0.96 & 1.04 \\
        0.77 & 0.99 & 0.84 & 1.08 & 1.12 & 0.78 & 0.95 & 0.93 & 1.09 & 0.92 \\
        1.00 & 0.92 & 1.02 & 0.90 & 0.87 & 0.85 & 1.03 & 1.04 & 0.92 & 1.07
    \end{matrix}
\end{equation*}

Supongamos que el peso de un paquete elegido al azar es una variable aleatoria $X \sim \mathcal{N}(\mu,\sigma^2)$

Al elegir $n$ paquetes tenemos: Sea $X_{1},X_{2},\dots,X_{n}$ i.i.d. $X_{i} \sim \mathcal{N}(\mu,\,\sigma^{2})$
Estimador de los momentos de $\mu$ de orden 1:
\begin{equation*}
    \frac{1}{n} \sum_{i=1}^n X_{i} = E_{\widehat{\mu}}(X) = \widehat{\mu}
\end{equation*}
Con estos datos la estimacion que se obtiene es $\widehat{\mu}_{obs} = 0.97$

\subsection{Ejercicio 2 - Distribucion Exponencial}
Una fabrica de lamparas sabe que el tiempo de vida, en dias, de las lamparas que fabrica, sigue una distribucion $Exp(\theta)$.
Obtener una formula para estimar $\theta$ a partir de una muestra aleatoria $X_{1}\dots X_{n}$

Antes de probar las lamparas no sabemos cuanto durara cada una. Asi la duracion de la primera puede ser considerada
una v.a $X_{1}$. la segunda una v.a $X_{2}$, etc. 

\begin{equation*}
    X_{1}, X_{2}, \dots, X_{n} \sim Exp(\theta)
\end{equation*}

Para hallar el estimador de momentos de $\theta$, hay que despejar $\widehat{\theta}$ de
\begin{equation*}
    \frac{1}{n} \sum_{i=1}^n X_{i} = E_{\widehat{\theta}}(X_{1})
\end{equation*}
\begin{equation*}
    \frac{1}{n} \sum_{i=1}^n X_{i} = \frac{1}{\mathcal{\theta}} \Rightarrow \mathcal{\theta} = \frac{1}{\widehat{X}}
\end{equation*}

\subsection{Ejercicio 3 - Distribucion uniforme}
\begin{equation*}
    X_{1}, X_{2}, \dots , X_{n} \sim \mathcal{U}{0, \theta}
\end{equation*}
Hay que despejar $\theta$ de
\begin{equation*}
    \frac{1}{n} \sum_{i=1}^n X_{i} = E_{\widehat{\theta}}(X_{1})
\end{equation*}
\begin{equation*}
    \frac{1}{n} \sum_{i=1}^n X_{i} = \frac{\widehat{\theta}}{2} \Rightarrow \mathcal{\theta} = \frac{1}{2n}\sum_{i=1}^{n} X_{i}
\end{equation*}

\subsection{Ejercicio 4 - Estimacion de ambos parametros de la normal}
Supongamos que tenemos una muestra aleatoria $X_{1}, X_{2},\dots,X_{n} \sim \mathcal{N}(\mu, \sigma^2)$ \\
Se tiene que $E_{\mu, \sigma^2}(X) = \mu$ y $E_{\mu, \sigma^2}(X^2) = \mu^2 + \sigma^2$
Para encontrar el estimador de momentos de $\mu$ y $\sigma$ hay que resolver el sistema

\begin{equation*}
    \frac{1}{n} \sum_{i=1}^n X_{i} = E_{\widehat{\mu}, \widehat{\sigma}^2}(X_{1}) = \widehat{\mu}
\end{equation*}
\begin{equation*}
    \frac{1}{n} \sum_{i=1}^n X_{i}^2 = E_{\widehat{\mu}, \widehat{\sigma}^2}(X_{1}^2) = \widehat{\mu}^2 + \widehat{\sigma}^2
\end{equation*}

\begin{equation*}
    \begin{cases}
        \frac{1}{n} \sum_{i=1}^n X_{i} = \widehat{\mu}
        \\
        \frac{1}{n} \sum_{i=1}^n X_{i}^2 = \widehat{\mu}^2 + \widehat{\sigma}^2
        \end{cases}
\end{equation*}
\begin{equation*}
    \widehat{\mu} = \frac{1}{n} \sum_{i=1}^n X_{i}
\end{equation*}
\begin{equation*}
    \widehat{\sigma}^2 = \frac{1}{n} \sum_{i=1}^n X_{i}^2 - (\frac{1}{n} \sum_{i=1}^n X_{i})^2
\end{equation*}

\section{Ejemplos de estimacion puntual - Verosimilitud}
\subsection{Ejercicio 1 - $\mathcal{E}(\lambda):f(x,\lambda) = \lambda e^{-x\lambda}\mathcal{I}_{(0, \infty)}(x)$}
$X_{1},\dots,X_{n}$ v.a. i.i.d. $X_{i} \sim \mathcal{E}(\lambda), \lambda > 0$
\begin{equation*}
    L(\lambda; x) = \prod_{i=1}^n f(x_{i}, \lambda) = \prod_{i=1}^n \lambda e^{-x_{i}\lambda}\mathcal{I}_{(0,\infty)}(x_{i})
\end{equation*}
Si $x_{i}\geq 0 \forall i$
\begin{equation*}
    L(\lambda; x) = \lambda e^{-x_{i}\sum_{i=1}^n x_{i}}
\end{equation*}
Si consideramos $\log L$ resulta
\begin{equation*}
    l(\lambda;x) = n \log(\lambda)-\lambda \sum_{i=1}^n x_{i}
\end{equation*}
Derivando e igualando a 0 queda \\
$\frac{n}{\lambda} - \sum_{i=1}^n x_{i} = 0 \Rightarrow$ punto critico es $\frac{1}{\bar{x}_{n}}$, ver que maximiza \\
$\Rightarrow \widehat{\lambda} = \frac{1}{\bar{X}_{n}}$

\subsection{Ejercicio 2 - $X_{1},\dots,X_{n}$ v.a. i.i.d. $X_{i} \sim \mathcal{N} (\mu, 9)$, $f(x,\mu, 9) = \frac{1}{\sqrt{2\pi}}\frac{1}{3}e^{-\frac{1}{2}\frac{(x-\mu)^2}{9}}$ }
\begin{equation*}
    L(\mu, 9; x) = \prod_{i=1}^n f(x_{i}, \mu, 9) = \prod_{i=1}^n \frac{1}{\sqrt(2\pi)}\frac{1}{3}e^{-\frac{1}{2}\frac{(x_{i}-\mu)^2}{9}}
\end{equation*}
\begin{equation*}
    L(\mu, 9; x) = (\frac{1}{\sqrt(2\pi)})^n(\frac{1}{3})^n e^{-\frac{1}{2}\frac{\sum_{i=1}^n(x_{i}-\mu)^2}{9}}
\end{equation*}
Tomemos logatirmo
\begin{equation*}
    l(\mu, 9; x) = cte - \frac{1}{2}\frac{\sum_{i=1}^n(x_{i}-\mu)^2}{9}
\end{equation*}
Maximizar a $l(\mu, 9; x)$ como funcion de $\mu$ equivale a minimizar
\begin{equation*}
    h(\mu) = \sum_{i=1}^n (x_{i} - \mu)^2
\end{equation*}
Un par de clases aatras vimos que $h(\mu)$ se minimiza en $\bar{x}_{n}$
\begin{equation*}
    EMV de \ \mu : \widehat{\mu} = \bar{X}_{n}
\end{equation*}
\begin{equation*}
    L(\mu, 9; x) = \prod_{i=1}^n f(x_{i}, \mu, \sigma^2) = \prod_{i=1}^n \frac{1}{\sqrt(2\pi\sigma^2)}e^{-\frac{(x_{i}-\mu)^2}{2\sigma^2}}
\end{equation*}
\begin{equation*}
    L(\mu, 9; x) = \prod_{i=1}^n f(x_{i}, \mu, \sigma^2) = \prod_{i=1}^n \frac{1}{\sqrt(2\pi\sigma^2)}e^{-\sum_{i=1}^n \frac{(x_{i}-\mu)^2}{2\sigma^2}}
\end{equation*}
Tomando logatirmo y resolviendo las ecuaciones
\begin{equation*}
    \frac{\partial l (\mu, \sigma^2;x)}{\partial \mu} = 0 \ y \ \frac{\partial l (\mu, \sigma^2;x)}{\partial \sigma^2} = 0
\end{equation*}
se obtiene que los EMV de $\mu$ y $\sigma^2$ section
\begin{equation*}
    \widehat{\mu} = \bar{X}_{n} \ \ \ \ \widehat{\sigma}^2 = \frac{\sum_{i=1}^n ((X_{i} - \bar{X}_n))^2}{n}
\end{equation*}
Los estimadores de $\mu$ y $\sigma$ coinciden con los estimadores de momentos

\subsection{Ejercicio 2 - $\mathcal{U}(0,\theta):f(x,\theta) = \frac{1}{\theta}\mathcal{I}_{(0,\theta)}(x)$}
$X_{1},\dots,X_{n}$ v.a. i.i.d. $X_{i} \sim \mathcal{U}(0,\theta)$
\begin{equation*}
    f(x_{1},\dots,x_{n}, \theta) = \prod_{i=1}^n \frac{1}{\theta}I_{(0,\theta)}(x_{i}) = \frac{1}{\theta^n}\prod_{i=1}^n I_{(0,\theta)}(x_{i})
\end{equation*}
La indicadora vale 1 cuando todas las indicadoras valgan 1, de lo contrario, es 0
Por ende, la conjunta puede tomar 2 valores:
\begin{equation*}
    \begin{cases}
        \frac{1}{\theta^n} \ si \ 0<x_{i}<\theta \  \forall i
        \\
        0 \ en \ otro \ caso
        \end{cases}
\end{equation*}
\begin{equation*}
    \begin{cases}
        \frac{1}{\theta^n} \ si \ \theta > \max(x_{i})
        \\
        0 \ en \ otro \ caso
        \end{cases}
\end{equation*}
\end{document}
